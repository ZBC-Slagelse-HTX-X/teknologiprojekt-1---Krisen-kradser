\section{Problemanalyse}
Problemanalysen tager udgangspunkt i nøgleproblemet krisehåndtering jf projektbeskrivelsen, der kan findes som appendiks (\ref{appendiks-projektbeskrivels}).

\subsection{Problemtræ}
\begin{figure}[H]
    \centering
    \fbox{\includegraphics[width=\textwidth]{assets/section_3/Problemtræ.jpg}}
\end{figure}
\subsection{HV-modellen}
Ligedan er HV-modellen blevet brugt for at konkretisere, hvilke trin som tages for at udføre vores projekt.
\paragraph{Hvad?} - Det skal gøres overskueligt og konkret, hvordan man skal handle i en krisesituation.
\paragraph{Hvorfor?} - Det skal gøres, så man kan være beredt i en eventuel krisesituation. Fra et firmasynspunkt er der et stort udbyttepotentiale, da det må forventes, at folk værner om sit og sine næstes liv. Desuden er det sandsynligt, at eftersom beredskabstyrelsen har varslet information om kriseparathed \cite{krisemanual}, at man via en aftale med staten eventuelt kunne få et subsidium til udarbejdelsen af en sådan applikation, som heri benævnes.
\paragraph{Hvem?} - Projektet skal udarbejdes af en nyopstartet virksomhed, her fiktivt, "KEP". Se mere om "KEP"-størrelsen under kapitlet om distribution og praksisudførelse \ref{sec:distribution}.
\paragraph{Hvordan?} - Det skal gøres ved at lave et program, der gør overstående jf. "hvad?", via en moderne applikationsbyggeløsning, således at denne finde den gyldne vej imellem optimisering og kompatabilitet.