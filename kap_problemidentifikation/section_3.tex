\section{Problemidentifikation}\label{sec:problemidentifikation}
\subsection{Idegenerering}
\subsubsection{Mindmap}
Vi har valgt at lave et mindmap, da dette er en effektiv måde at generere ideer på, og få et overblik over hvilke emner der er relevante at beskæftige sig med.
\begin{figure}[H]
    \centering
    \fbox{\includegraphics[scale=0.25, angle=90]{assets/section_3/mindmap.jpg}}
    \caption{Viser vores mindmap}
\end{figure}

\subsubsection{Lyskurven}
Lyskurven \footnote{Systimebogen: Projektarbejdet > 3. Problemidentifikation > 3.3 Idesortering | ID 174 \cite{projektarbejdet}} er en metode, man kan bruge til at sortere ideerne i forhold til hvilke der er mest relevante. kort sagt benyttes Lyskurve modellen til at beslutte om en ide fortjener/kan svare sig at arbejde videre ved, dette sker ud fra de tre kategorier som du kan komme din ide ind i, den grønne, gule og røde. derfor Lyskurve modellen. Den blev under produktet brugt til at fokusere de ideer der blev skabt under idegeneringen.

\begin{table}[H]
    \centering
    \begin{tabular}{|c|c|c|}
        \hline
        Fødevarer & \textbf{Grøn} \\
        \hline
        Beskyttelsesrum & \textbf{Gul} \\
        \hline
        Nødstrøm & \textbf{Rød} \\
        \hline
    \end{tabular}
    \caption{Viser et meget abstrakt lyskurvediagram i form af en tabel; det vi anvendte}
\end{table}

\newpage

\subsubsection{Identificering af nøgleproblem}
Her anvendes spørgsmål til at sikre, at det umiddelbart interessant emne jf. lyskurven også rent faktisk er interessant:
\begin{displayquote}
\[...\] følgende spørgsmål:
\begin{enumerate}
    \item Hvorfor er det her interessant?
    \item Hvem er det interessant for?
    \item Er det noget, vi laver for vores egen fornøjelses skyld?
    \item Er det noget, som en bestemt gruppe i samfundet kan have gavn af, eller er det noget, der er til gavn for alle?
\end{enumerate}
\end{displayquote}
(Spørsmålene, som her gøres brug af, er hentet fra systimebogen\footnote{Systimebogen: Projektarbejdet >
3. Problemidentifikation > 3.3 identificering af nøgleproblem | ID 271 \cite{projektarbejdet}})

\vspace{1em}

Besvarelsen på disse spørgsmål ser således ud:
\begin{enumerate}
    \item Krisehåndtering er et interessant emne, da det har en stor betydning for alle i samfundet.
    \item Det er relavant for alle.
    \item Nej, ideen med produktet er at kunne hjælpe almindelige mennesker med at håndtere kriser, mindre som store.
    \item Produktet skulle kunne gavne alle, som kunne stå i en krisesituation.
\end{enumerate}

Her ses, at det mest interessante emne, jf. overstående, er krisehåndtering, da det både har en så stor relevans, herunder for mennesker i krisesituationer, og samfundet, og har en reel efterspørgsel udover den intrinsiske.