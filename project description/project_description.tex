%%% PREAMBLE
\documentclass[12pt, a4paper]{article}
%% Packages
% Layout
\usepackage{parskip} % Changes parskip from being indent-based to adding vertical space
\usepackage[margin=2.5cm]{geometry} % Adjust margins to comply with 2,5cm standard
\usepackage{float}
\usepackage{pdflscape}
\usepackage[danish]{translator}
\usepackage[danish]{babel} % Multi-lingual support - danish
\title{Projektbeskrivelse af projekt 1 "Krisen kradser" i faget Teknologi B}
\author{Alexander Knudsen, Andreas Jensen og Jeppe Bech}
\begin{document}
\maketitle
\section{Formalia}
\subsection{Gruppesammensætning}
Projektet, som heri beskrives, er udarbejdet af følgende personer: Alexander Knudsen, Andreas Jensen og Jeppe Bech (2. X).
\subsection{Tema}
Overordnet: Krisen kradser

Underemne: Fødevarer
\section{Produktbeskrivelse}
Produktet, kulminationen på projektet, er en offline softwareprogram med diverse delprogrammer til krisehåndtering, herunder en ernæringsberegner, der også indeholder modstykke, der foretager anbefalinger om afgrødevalg og allokering af areal til disse. 

Endvidere indeholder programmet også manualer med information til afhjælpning af diverse nødsituationer samt praktisk information vedrørende daglig ageren, fx istandsættelse og anvendelse af radio, reperation af bil, isolering af hus, et cetera.
\section{Tidsplan}
Tidsplanen og tidsstyringen foregår via GitHubs indbyggede funktionlaitet, dog tilstræbes det som udgangspunkt, at projketet foregår over fem faser, dog er dette fleksibelt og kan korrigeres undervejs:
\begin{enumerate}
    \item Dataindsamling og skitseudarbejdelse (Projektesgrundlag) - varighed (1 uge)
    \item Brugeroverflade samt udarbejdelse af hardwarepersona (Produktudvikling) - varighed (3 dage)
    \item Ernærings- og afgrødeberegner (Produktudvikling) - varighed (1 uge og 4 dage)
    \item Manualuddrag samt bogtilvalg til henvisning, finpudsning og evt. \textit{visualiser} - varighed (1 uge)
    \item Evt. distribution og afrappotering - varighed (1 uge)
\end{enumerate} 
\end{document}