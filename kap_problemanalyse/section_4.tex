\section{Problemanalyse}
Problemanalysen tager udgangspunkt i nøgleproblemet krisehåndtering jf projektbeskrivelsen, der kan findes som appendiks (\ref{apx:projektbeskrivels}) samt kan man læse kapitlet om problemidentifikation (\ref{sec:problemidentifikation}).

\subsection{Problemtræ}
\begin{figure}[H]
    \centering
    \fbox{\includegraphics[width=\textwidth]{assets/section_3/Problemtræ.jpg}}
    \caption{Viser vores problemtræ}
\end{figure}

\subsection{Kvalitativ metode}
Beredskabstyrelsen har udarbejdet og sendt en opfordring til alle danskere om kriseparathed, hvor de tydeligøre vigtigheden i at være beredt på at kunne håndtere en eventuel krisesituation.
De beskriver i denne, hvordan danskerne skal kunne klare sig selv i 3 døgn, og hvad der skal til for at dette er muligt.
I vores husstande, i det private, er disse opfordringer blevet taget meget seriøst. Men hvad gør man, hvis man løber tør for drikkevand, mad eller lignende? Man kan nødvendivis ikke tilgå hjælp via internettet.
Vi så her en god mulighed, med vores appudviklingsenskaber, at kunne løse dette problem med at lave en app, der kan hjælpe med at håndtere kriser, både forberedene, på kort sigt og på lang sigt.

\subsection{HV-modellen}
Ligedan er HV-modellen blevet brugt for at konkretisere, hvilke trin som tages for at udføre vores projekt.
\paragraph{Hvad?} - Det skal gøres overskueligt og konkret, hvordan man skal handle i en krisesituation.
\paragraph{Hvorfor?} - Det skal gøres, så man kan være beredt i en eventuel krisesituation. Fra et firmasynspunkt er der et stort udbyttepotentiale, da det må forventes, at folk værner om sit og sine næstes liv. Desuden er det sandsynligt, at eftersom beredskabstyrelsen har varslet information om kriseparathed \cite{krisemanual}, at man via en aftale med staten eventuelt kunne få et subsidium til udarbejdelsen af en sådan applikation, som heri benævnes.
\paragraph{Hvem?} - Projektet skal udarbejdes af en nyopstartet virksomhed, her fiktivt, "KEP". Se mere om "KEP"-størrelsen under kapitlet om distribution og praksisudførelse \ref{sec:distribution}.
\paragraph{Hvordan?} - Det skal gøres ved at lave et program, der gør overstående jf. "hvad?", via en moderne applikationsbyggeløsning, således at denne finde den gyldne vej imellem optimisering og kompatabilitet.