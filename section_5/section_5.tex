\section{Produktprincip}
\subsection{Målgruppe}
Det tilsigtes, at produktet skal være tilgængeligt for alle danskere, da kriser ikke diskriminerer. Det betyder i praksis, at personer, der har diverse handicap, skal være i stand til at kunne produktet, herunder folk, som er ord- eller farveblinde.
\subsection{Kravspecifikation}
Da det må forventes, at produktet skal kunne anvnedes i tilfælde af et nedbrud af internettet, herfor skal produktet kunne fungere uden internetforbindelse dvs. applikationen skal være offline--og da det er planen, at appen skal indeholde videoer, skal disse ligedan nedhentes.
Desuden skal produktet være simpelt og intuitivt at forstå, herunder have ordforklaringer og selve brugergrænsefladen skal være let at navigere rund i og følge nuværende UI-standarder.
Produktet skal ligedan være et kompromis mellem design og letvægtighed, herved forstås der, at appen ikke er resursekrævende, således at selv ældre hardware kan tilgå appen.

Hermed på listeform:
\begin{itemize}
    \item Offlinefunktionalitet
    \item Letvægtighed
    \item Simpel brugergrænseflade samt ordforklaringer
\end{itemize}

\subsection{Konkurrenter}
Der er ikke umiddelbart nogen decideret konkurrent til produktet, især ikke på det danske marked, som henvendes til, men der et utal af videoer og guides om overlevelse i det vilde og diverse beregnere på internettet, men her adskiller det forslået produkt sig ved at være tilgængeligt uagtet internetforbindelse, mere omfattende og tilgængeligt på dansk.